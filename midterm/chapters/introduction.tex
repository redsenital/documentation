\chapter{ Introduction}
Web applications today play a central role in handling sensitive information for individuals and organizations. Despite advances in secure software engineering, these systems continue to face long-standing vulnerabilities. Cross-Site Scripting (XSS) remains one of the most persistent and widely exploited threats and consistently appears in the OWASP Top Ten rankings. XSS attacks occur when adversaries inject malicious scripting code into web content that is later delivered to end users, potentially enabling credential theft, unauthorized session access, malware distribution, or manipulation of displayed data.\\                              Modern web pages integrate multiple languages,including HTML, CSS, JavaScript, and SVG, within the same document structure. Because each injection point follows different syntactic and execution rules, attackers must tailor payloads to the specific context in which the code will run. Research has shown that a payload effective inside a script block may fail entirely when placed within an HTML attribute. This context sensitivity is a major reason why XSS detection and prevention remain difficult even after decades of research.\\                       
\section{ Problem Statement}
Although XSS scanners and WAFs can identify many basic vulnerabilities, they lack the adaptive intelligence required to generate valid and context-aware payloads. Their reliance on predefined signatures limits their ability to detect new, obfuscated, or context-specific attack vectors. There is a clear need for a system that can automatically generate intelligent, adaptable, and context-sensitive XSS payloads using modern machine learning techniques, thereby improving the depth and reliability of security testing.\\  These challenges have motivated the use of artificial intelligence and machine learning to support offensive security research. Recent work demonstrates that generative models, especially transformer-based architectures, are capable of producing diverse and novel XSS payloads that surpass human-crafted examples. Systems such as GenXSS and fine-tuned language models like GPT-2 and CodeT5 show promising results in automating adversarial payload creation, thereby increasing coverage and uncovering weaknesses that rule-based tools fail to detect.
\section{ Objectives}
\begin{enumerate}
    \item To develop a machine learning–based system capable of generating context-aware XSS payloads.
    \item Integrate payload generation with a modular testing framework.
\end{enumerate}
\section{Project Feasibility}
The Red Sentinel project is feasible due to the maturity of ML frameworks, availability of datasets, and growing interest in automated offensive security methods.
\subsection{Technical Feasibility}
\begin{enumerate}
    \item Modern ML frameworks such as TensorFlow and PyTorch support transformer architectures and byte-level tokenizers, making implementation highly achievable.
    \item Hardware requirements for training a medium-sized transformer model are modest and achievable using consumer GPUs or cloud resources.
\end{enumerate}
\subsection{Operational Feasibility}
\begin{enumerate}
    \item Security teams can incorporate Red Sentinel into existing testing workflows without major changes.
    \item The modular design supports iterative development and ease of maintenance.
    \item The system reduces manual effort by automating payload creation, improving operational efficiency for developers and penetration testers.
\end{enumerate}
\subsection{Economic Feasibility}
\begin{enumerate}
    \item Costs are limited primarily to training infrastructure and optional dataset acquisition.
    \item Open-source ML libraries and penetration-testing tools reduce development expense.
    \item Cloud-based compute resources can be used only when necessary, minimizing expenses.
\end{enumerate}
\subsection{Sociocultural Feasibility}
\begin{enumerate}
    \item The project aligns with growing societal emphasis on cybersecurity and ethical hacking
    \item The project allows for automated check of vulnerabilities by untrained personnel.
\end{enumerate}
\subsection{Legal Feasibility}
\begin{enumerate}
    \item The system must be used strictly for authorized security testing to comply with cybercrime and computer misuse laws.

    \item No proprietary or personal data is required for the model, reducing legal risks.
\end{enumerate}